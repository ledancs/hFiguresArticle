%% BioMed_Central_Tex_Template_v1.06
%%                                      %
%  bmc_article.tex            ver: 1.06 %
%                                       %

%%IMPORTANT: do not delete the first line of this template
%%It must be present to enable the BMC Submission system to
%%recognise this template!!

%%%%%%%%%%%%%%%%%%%%%%%%%%%%%%%%%%%%%%%%%
%%                                     %%
%%  LaTeX template for BioMed Central  %%
%%     journal article submissions     %%
%%                                     %%
%%          <8 June 2012>              %%
%%                                     %%
%%                                     %%
%%%%%%%%%%%%%%%%%%%%%%%%%%%%%%%%%%%%%%%%%


%%%%%%%%%%%%%%%%%%%%%%%%%%%%%%%%%%%%%%%%%%%%%%%%%%%%%%%%%%%%%%%%%%%%%
%%                                                                 %%
%% For instructions on how to fill out this Tex template           %%
%% document please refer to Readme.html and the instructions for   %%
%% authors page on the biomed central website                      %%
%% http://www.biomedcentral.com/info/authors/                      %%
%%                                                                 %%
%% Please do not use \input{...} to include other tex files.       %%
%% Submit your LaTeX manuscript as one .tex document.              %%
%%                                                                 %%
%% All additional figures and files should be attached             %%
%% separately and not embedded in the \TeX\ document itself.       %%
%%                                                                 %%
%% BioMed Central currently use the MikTex distribution of         %%
%% TeX for Windows) of TeX and LaTeX.  This is available from      %%
%% http://www.miktex.org                                           %%
%%                                                                 %%
%%%%%%%%%%%%%%%%%%%%%%%%%%%%%%%%%%%%%%%%%%%%%%%%%%%%%%%%%%%%%%%%%%%%%

%%% additional documentclass options:
%  [doublespacing]
%  [linenumbers]   - put the line numbers on margins

%%% loading packages, author definitions

\documentclass[twocolumn]{bmcart}% uncomment this for twocolumn layout and comment line below
%\documentclass{bmcart}

%%% Load packages
%\usepackage{amsthm,amsmath}
\RequirePackage{natbib}
\RequirePackage{hyperref}
\usepackage[utf8]{inputenc} %unicode support
\usepackage{pifont}
%\usepackage[applemac]{inputenc} %applemac support if unicode package fails
%\usepackage[latin1]{inputenc} %UNIX support if unicode package fails
\usepackage{multirow}
% Check : http://tex.stackexchange.com/questions/12703/how-to-create-fixed-width-table-columns-with-text-raggedright-centered-raggedlef
\usepackage{array}
\newcolumntype{L}[1]{>{\raggedright\let\newline\\\arraybackslash\hspace{0pt}}m{#1}}
\newcolumntype{C}[1]{>{\centering\let\newline\\\arraybackslash\hspace{0pt}}m{#1}}
\newcolumntype{R}[1]{>{\raggedleft\let\newline\\\arraybackslash\hspace{0pt}}m{#1}}
\usepackage{morefloats}

%%%%%%%%%%%%%%%%%%%%%%%%%%%%%%%%%%%%%%%%%%%%%%%%%
%%                                             %%
%%  If you wish to display your graphics for   %%
%%  your own use using includegraphic or       %%
%%  includegraphics, then comment out the      %%
%%  following two lines of code.               %%
%%  NB: These line *must* be included when     %%
%%  submitting to BMC.                         %%
%%  All figure files must be submitted as      %%
%%  separate graphics through the BMC          %%
%%  submission process, not included in the    %%
%%  submitted article.                         %%
%%                                             %%
%%%%%%%%%%%%%%%%%%%%%%%%%%%%%%%%%%%%%%%%%%%%%%%%%


\def\includegraphic{}
\def\includegraphics{}



%%% Put your definitions there:
\startlocaldefs
\endlocaldefs


%%% Begin ...
\begin{document}

%%% Start of article front matter
\begin{frontmatter}

\begin{fmbox}
\dochead{Research}

%%%%%%%%%%%%%%%%%%%%%%%%%%%%%%%%%%%%%%%%%%%%%%
%%                                          %%
%% Enter the title of your article here     %%
%%                                          %%
%%%%%%%%%%%%%%%%%%%%%%%%%%%%%%%%%%%%%%%%%%%%%%

\title{Health Figures: An Open Source JavaScript Library for Health Data Visualization}

%%%%%%%%%%%%%%%%%%%%%%%%%%%%%%%%%%%%%%%%%%%%%%
%%                                          %%
%% Enter the authors here                   %%
%%                                          %%
%% Specify information, if available,       %%
%% in the form:                             %%
%%   <key>={<id1>,<id2>}                    %%
%%   <key>=                                 %%
%% Comment or delete the keys which are     %%
%% not used. Repeat \author command as much %%
%% as required.                             %%
%%                                          %%
%%%%%%%%%%%%%%%%%%%%%%%%%%%%%%%%%%%%%%%%%%%%%%

\author[
   addressref={aff1},                   % id's of addresses, e.g. {aff1,aff2}
   corref={aff1},                       % id of corresponding address, if any
   %noteref={n1},                        % id's of article notes, if any
   email={andres.ledesma@tut.fi}   % email address
]{\inits{JE}\fnm{Andres} \snm{Ledesma}}
\author[
   addressref={aff1},
   %noteref={n1},
   email={hannu.nieminen.o@tut.fi}
]{\inits{JRS}\fnm{Hannu} \snm{Nieminen}}
\author[
   addressref={aff1},
   %noteref={n1},
   email={mohammed.al-musawi@student.tut.fi}
]{\inits{JRS}\fnm{Mohammed} \snm{Al-Musawi}}

%%%%%%%%%%%%%%%%%%%%%%%%%%%%%%%%%%%%%%%%%%%%%%
%%                                          %%
%% Enter the authors' addresses here        %%
%%                                          %%
%% Repeat \address commands as much as      %%
%% required.                                %%
%%                                          %%
%%%%%%%%%%%%%%%%%%%%%%%%%%%%%%%%%%%%%%%%%%%%%%

\address[id=aff1]{%                           % unique id
  \orgname{Department of Signal Processing, Tampere University of Technology}, % university, etc
  \street{Korkeakoulunkatu 10},                     %
  \postcode{33720}                                % post or zip code
  \city{Tampere},                              % city
  \cny{Finland}                                    % country
}
%\address[id=aff2]{%
%  \orgname{Marine Ecology Department, Institute of Marine Sciences Kiel},
%  \street{D\"{u}sternbrooker Weg 20},
%  \postcode{24105}
%  \city{Kiel},
%  \cny{Germany}
%}

%%%%%%%%%%%%%%%%%%%%%%%%%%%%%%%%%%%%%%%%%%%%%%
%%                                          %%
%% Enter short notes here                   %%
%%                                          %%
%% Short notes will be after addresses      %%
%% on first page.                           %%
%%                                          %%
%%%%%%%%%%%%%%%%%%%%%%%%%%%%%%%%%%%%%%%%%%%%%%

\begin{artnotes}
%\note{Sample of title note}     % note to the article
%\note[id=n1]{Personal Health Informatics Research Group} % note, connected to author
\end{artnotes}

%\end{fmbox}% comment this for two column layout

%%%%%%%%%%%%%%%%%%%%%%%%%%%%%%%%%%%%%%%%%%%%%%
%%                                          %%
%% The Abstract begins here                 %%
%%                                          %%
%% Please refer to the Instructions for     %%
%% authors on http://www.biomedcentral.com  %%
%% and include the section headings         %%
%% accordingly for your article type.       %%
%%                                          %%
%%%%%%%%%%%%%%%%%%%%%%%%%%%%%%%%%%%%%%%%%%%%%%

\begin{abstractbox}

\begin{abstract} % abstract

\parttitle{Background} %if any
The way we look at data has a great impact on how we can understand it, particularly when the data is related to health and wellness. Due to the increased use of self-tracking devices and the ongoing shift towards preventive medicine, better understanding of our health data is an important part of improving the general welfare of the citizens. Electronic Health Records, self-tracking devices and mobile applications provide a rich variety of data but it often becomes complex and difficult to understand. Recent insight-based studies have shown that normalized visualizations of health data can provides a better understanding on health behaviors and their relationship with the overall wellness. Based on the hGraph visualization developed earlier, we implemented an open source library that addresses some usability issues found on the hGraph and extends its functionality to further enhance health data understanding and analysis including a time-based visualization of the data. We named this library hFigures. The hFigures library provides features to improve the understanding of health data. The library enables comparison of multiple health measurements organizing the measurements in groups separated by a structured layout. The purpose is to enable a visual representation of the evolution of health measurements in a comprehensive and holistic manner. 

\parttitle{Results} %if any
We built an application using the hFigures library for health data visualization of a modelled patient in a health coaching program. We performed a user evaluation with Heuristic Evaluation, Controlled User Testing and Usability Questionnaires. The results were satisfactory, users found the library informative and usable with a high score in user satisfaction.

\parttitle{Conclusions} %if any
We developed hFigures, an open source library for visualizing a holistic, normalized and comprehensive graphical representation of health data. The idea is based on the concept of the hGraph but it provides additional key features, including a comparison of multiple health measurements over time. We evaluated the library in an application and obtained positive results.

\end{abstract}

%%%%%%%%%%%%%%%%%%%%%%%%%%%%%%%%%%%%%%%%%%%%%%
%%                                          %%
%% The keywords begin here                  %%
%%                                          %%
%% Put each keyword in separate \kwd{}.     %%
%%                                          %%
%%%%%%%%%%%%%%%%%%%%%%%%%%%%%%%%%%%%%%%%%%%%%%

\begin{keyword}
\kwd{data visualization}
\kwd{health data}
\kwd{health informatics}
\kwd{javascript}
\end{keyword}

% MSC classifications codes, if any
%\begin{keyword}[class=AMS]
%\kwd[Primary ]{}
%\kwd{}
%\kwd[; secondary ]{}
%\end{keyword}

\end{abstractbox}
%
\end{fmbox}% uncomment this for twcolumn layout

\end{frontmatter}

%%%%%%%%%%%%%%%%%%%%%%%%%%%%%%%%%%%%%%%%%%%%%%
%%                                          %%
%% The Main Body begins here                %%
%%                                          %%
%% Please refer to the instructions for     %%
%% authors on:                              %%
%% http://www.biomedcentral.com/info/authors%%
%% and include the section headings         %%
%% accordingly for your article type.       %%
%%                                          %%
%% See the Results and Discussion section   %%
%% for details on how to create sub-sections%%
%%                                          %%
%% use \cite{...} to cite references        %%
%%  \cite{koon} and                         %%
%%  \cite{oreg,khar,zvai,xjon,schn,pond}    %%
%%  \nocite{smith,marg,hunn,advi,koha,mouse}%%
%%                                          %%
%%%%%%%%%%%%%%%%%%%%%%%%%%%%%%%%%%%%%%%%%%%%%%

%%%%%%%%%%%%%%%%%%%%%%%%% start of article main body
% <put your article body there>

%%%%%%%%%%%%%%%%
%% Background %%

\section*{Introduction}
The ongoing shift from reactive to preventive medicine requires that citizens have the skills and means to take an active role in developing and maintaining their wellness. Individuals have greater access to self-tracking devices and applications, from sports tracking application to genome sequence analysis. These services and devices produce large amounts of data. In addition, Electronic Health Records are increasingly replacing paper records in hospitals and clinics around the world. The combination of these large and heterogeneous data sources can provide a ``predictive, preventive, personalized and participatory'' ecosystem to the benefit of the general welfare \cite{Hood2012613}.

To better understand our health, we need to combine heterogeneous data sources and present the information to the user with a comprehensive and holistic approach. In order to accomplish this, ``new hybrid professionals'' need to integrate health information technologies and visualization design \cite{lesselroth2011data}. The goal is to provide tools for individuals to take better decisions regarding their health. Similarly, doctors and other medical experts need tools and solutions for getting a comprehensive view of the patients health, combining togehter patient's own measurements and clinical data.

Graphical perception is the ability of an individual to decode the information displayed as graphical objects \cite{baird1978fundamentals} and it is a widely researched field \cite{cleveland1984graphical}. Graphical perception affect how we understand visualized information. In the context of health data, it remains a challenge to design graphical representations for non-medical experts. Graphical representations in this context should enhance the users' ability to understand their health situation and take informed decisions. With this ``deep'' understanding on the health situation, individuals can move towards healthier behaviors.

An innovative approach for health data visualization is the Health Graph (hGraph), released publicly by MITRE corporation \cite{follett2012hgraph}. Based on a previous study \cite{EMBCLedNiem}, we found that the hGraph enhances deep understanding of health data and enables the user to create meaningful health insights based on the interrelationships between the measurements. However, the hGraph shows a static overview of a persons wellness. That is, the values of multiple measurements at one point in time. While this visualization is useful for the purpose of understanding with a quick glance the overall situation, it lacks features such as a clear division of measurements according to their category, a distribution of the labels to avoid clutter and other aspects. In our previous study \cite{EMBCLedNiem}, users provided feedback on what the hGraph was missing to help them understand in an easier and more comprehensive manner the data in the experiment.

We used the feedback from the study along with a user-centered design approach to build our visualization library. We added features that users considered helpful in understanding the health data better. In order to build a visualization tool that enables users to take informed decisions regarding their health habits, we addressed these requests and extended the original design and functionality of the hGraph. We open sourced our library under the name Health Figures, or hFigures for short. 

This article presents the research background on health data visualizations and an overview on the hGraph. The article details the design and the implementation implications of the user feedback in building the hFigures library. We summarize the features we implemented and discuss how this library can be used in Personal Health Informatics and in Health Care processes.

% Andres has reviewd and edited the introduction.

\section*{Background}

\subsection*{Related Work}

% This needs to flow a bit better... DONE!

Visualization tools have mostly focused on Healthcare Information Systems and Electronic Health Records (EHR) \cite{lesselroth2011data}. For instance, TimeLine is a software developed to retrieve data from several sources and presented in a hierarchical and timeline based structure where clinicians can browse chronologically through existing EHRs including MRI \cite{bui2007information}.

Additionally, the growing market for mobile health applications (mHealth) have drawn the attention of researchers, developers and investors \cite{mHealth2014}. These applications provide large volumes of personal health data. While the market and demand are expected to grow, the use of the data has the potential to contribute to a better understanding of our health.

Goetze \cite{goetz2015} demonstrated the impact of data visualization as means to represent health data in a comprehensive manner. He conducted a project that redesigned laboratory test results from numerical tables into colored graphics. He demonstrated that the patients were able to understand better their health situation when presented with the new designs.

Data integration for health monitoring as a Big Data process for personalized medicine has been approached by \cite{7072838}. The visualization of this information uses traditional bar and pie charts to report to the user a historical view of a variety of data including mental, social, physical aspects. The novelty of this work is the integration of heterogeneous data sources while the presentation of the information was done following existing graphical representations.

An extensive choice of graphical representation is listed and explained by S. Few \cite{few2006information}. These techniques have been studied and used widely among researchers and individuals alike. Examples include: bar, stacked bar, line and bullet graphs. These visualizations can be combined to provide a personalized wellness indicator system, as proposed by Soomlek and Benedicenti \cite{Soomlek:2013:AWI:2532685.2532686}.

DeRidder \textit{et al.} developed a combined approach that retrieves data from Personal Health Records (PHRs), and presents them to individual patients using a ``3D medical graphical avatar'' \cite{de2013web}. The solution is built using HTML5 and WebGL to render 3D graphics using the web browser. Patients can browse ``regions of interests'' on their avatar and explore further the information contained in EHRs as well as in PHRs.

However, as stated by Shneiderman \textit{et al.} \cite{shneiderman2013improving}, new visual representations are needed for ``systematic yet flexible visual analytics processes''. We present an existing tentative solution known as the hGraph, released publicly by MITRE corporation. We describe its main advantages and how they can address these challenges. In a previous study we identified possible improvements and based on our own implementation we addressed these issues and extended the features of the hGraph to better address complex data sources.

\subsection*{The Health Graph}

The health graph, or hGraph was developed by MITRE Corporation and released to the public in 2010 under the Apache v2.0 license. The design intention of the hGraph is to facilitate the graphical representation and understanding of health data. The data can come from a wide-range of sources such as laboratory tests, physical activity, nutrition, sleep monitors and other sources. The domain of this visualization technique includes personal health informatics, EHR and Personal Health Record (PHR) visualization \cite{follett2012hgraph}.

\subsubsection*{Design}
% refer to the figure to improve clarity
The design consists of a circular space with an area defined by to circumferences. The area represents the minimum and maximum recommended values for a given measurement. For instance, the minimum and maximum fat percentage of a person in a given age. The measurements are represented as circles and their position in the circular space represents how far or close they are from the recommended values. The position is normalized according to the recommended values. The values are distributed in a circular space.

\paragraph*{Graph} A graph is formed by joining the data points around the circular area. This polygon or graph reveals a pattern and its shape provides a quick overview of the general situation of all the values and how they deviate from the recommendations.

\paragraph*{Highlight Deviations} The hGraph design highlights values outside of the recommendation by using the red color on the data points and by modifying the shape of the graph.

\paragraph*{Rationale} The rationale of the hGraph is that if the same measurements are plotted in the same order for various cases, then the graph patterns can reveal similar shapes associated with certain health conditions.

\subsubsection*{Web-based Solution}

Web-based solutions for data visualization provide flexibility, as they can be accessed by any web browser, either from mobile devices or personal computers. The hGraph uses a web approach via HyperText Markup Language (HTML) and Scalable Vector Graphics (SVG). The programming language of the library is JavaScript and is built using the Data-Driven Documents library. Data-Driven Documents (D3.js) library provides free access to the Document Object Model (DOM), which is the substrate that enables the programmer to interface with the graphical representations in a web browser \cite{bostock2011d3}.

\subsubsection*{hGraph as an Insightful Visualization}
We conducted a study based on the approach proposed by C. North \cite{north2006toward} which compares visualizations based on how well users can derive meaningful insights from the data based on the visualization. We evaluated the hGraph visualization along with four alternatives using the Graphical Perception Framework proposed by Cleveland McGill \cite{cleveland1984graphical}. The evaluation followed the insight-based methodology similar to other experiments \cite{saraiya2004evaluation}. Additionally we interviewed the hGraph users on any issues and problems they experienced in understanding the graphical representation of the data. The experiment results are discussed in \cite{EMBCLedNiem}. The objective of the experiment was to compare different visualizations and determine how they can enable users to better understand their overall health situation, as well as the possible causes behind health issues. Out of the compared solutions, the hGraph was found to be the most effective solutions for creating meaningful insights and to help users to better understand the data.

% HANNU: This next chapter starting with Data points... need sclarification, the meaning is not totally clear.

% Andres: Yes agreed! I have rewritten the whole paragraph to include more examples and remove the use of the term "data points".

The hGraph scales to large amounts of data entries since the measurements, represented as circles, are distributed among a circumference. Other visualizations, such as bar charts, lines or scattered plots, can quickly take a large space when plotting large datasets. As more measurements are added, the circular space utilized in the hGraph will eventually stretch to accommodate large datasets, but the distribution stays uniform along the circle and the space required is still less than with other alternative visualizations.

Figure \ref{figureHGraphZoom} is an example of the hGraph visualization. The figure was extracted as a snapshot as the library generates an SVG document structure that cannot be exported outside the browser. We address this issue in the next section of the article. The hGraph hides the measurements when the zoom level is low, meaning that the user has zoomed out. The shape is an average of the deviation of each measurement under the same category. When the user zooms in, the details are revealed and the rest of the information becomes visible, that is the numerical values and positions (with respect to the recommendation) of the measurements.

\section*{Implementation} 

Throughout the design and implementation of our library, we adopted a user-centered approach. Invited users gave us continuous feedback on how they would understand the data better and what features would improve their experience with the library. Based on iterative design sessions and implementation of prototypes we developed the features based on how they enhance the usability and utility of the library. We named our library Health Figures (hFigures) based on the fact that we obtained our inspiration from the hGraph but eventually extended the library to include multiple graphs giving the notion of several intertwined figures.

This chapter explains the implementation process in our approach based on user feedback and on previous experiments \cite{EMBCLedNiem}. 


\subsection*{Data Source}

We use a JSON (JavaScript Object Notation) format to read the data, in which the measurements are grouped according to their categories. The groups contain an array of samples, which represent the values obtained from a measurement (steps per day, cholesterol, triglycerides, blood sugar or depression level using \cite{poutanen2010validity}). The samples contain a timestamp in Unix Epoch format and the value of the measurement. The Unix Epoch format is the number of seconds since the first of January 1970, Greenwich Meridian Time (GMT). An example of the data source is in figure \ref{codeDataSource}, it shows the first measurement of the group "Blood Pressure" which in this case is comprised by Systolic and Diastolic measurements and each of them have two samples taken at two particular times, Friday 9th of January 2015 at 10:10:24 GMT (1420798224 --- seconds) and Thursday 12th of February 2015 at 12:05:20 GMT (1423742720 --- seconds).

\subsection*{SVG Document Export}

The SVG document structure we designed in our implementation can be exported to a separate file outside the web browser. The short-term objective is to build a tool-kit that enables researchers to visualize their data with our implementation so they can use the generated SVG file in articles, posters, presentations or other applications. For instance, figure \ref{figureHFiguresSimple} has been exported as an SVG document from the browser into this article. SVG export is possible due to the rendering of our algorithm which does not depend on JavaScript or Cascade Style Sheet (CSS) styling properties to produce a finalized document. The library build the entire image as a stand-alone document. The hGraph library unfortunately does not produce a complete document but instead depends on CSS and JavaScript code to make the image visible.

\subsection*{Constant Graph Shape}

Figure \ref{figureHGraphZoom} shows an hGraph example and figure \ref{figureHFiguresSimple} shows an hFigures example. The hGraph computes the average of the deviation of the measurements in order to show the polygon or graph, as a representation of the overall health assessment. However, some measurement might deviate towards a lower value and while others towards a higher one, thus the average position would be roughly the middle recommended area. For this reason, hFigures does not change the shape of the graph if the user zooms in or out. Showing and hiding the measurement labels is the only reaction to the zooming events from the user at the moment. This avoids clutter when the user wished to have a quick glance at the picture but keeps the graph with the same shape.

\subsection*{Layout Construction}

The measurement groups are represented using a circular layout divided in sectors. The goal is to have a clear division between the groups as they represent the different aspects of the overall health. hFigures uses the d3 pie chart layout and modifies the data source provided to the layout. All the measurements have the same numerical value and at the end of the group, we insert an extra value in order to leave a blank space between the circular area sectors. The result is visible in the hFigures example shown in figure \ref{figureHFiguresSimple} and the code that produced this visualization is in figure \ref{codeLayoutConstruction}. The pie layout constructs the sectors of the circular area based on a data source. When we provide an array of numbers, the layout uses the numbers to calculate the proportions of the area. In order to achieve the layout construction that we have designed, the array has the same constant number multiple times, the number of measurements plus an additional number for each group. 

\subsection*{Color-coded Entries}

The data source can contain additional sets of value ranges. For instance a warning range of values can let the users know when a value has reached a level that requires attention but has not yet reached a critical point. We followed the users' feedback that recommends a traffic light-based approach. The green color means that the values are within the recommended, yellow suggest a warning or follow-up action needed and the red indicated a critical threshold has been passed.

In the implementation, the library verifies if these additional ranges are present in the measurement definition. In order to verify if the property of the object exists, JavaScript provides a qualifier method, \texttt{typeof}. The returned value must be compared with the keyword definition for properties that are not present in an object, the keyword \texttt{undefined} has been suggested by Mozilla Developer Network \cite{MDNJS}, a highly reputable source for Web development. The code is shown in figure \ref{codeColorCoded}.

\subsection*{Multiple Graph}

The dataset is structured as a set of measurements where each has its own collection of samples. In order to compare the evolution of these measurements, the hFigures library allows the graphical representation of any number of samples. The result is a set of graphs or polygons overlapping or stacking with each other. In order to differentiate them, we use a lighter set of colors so that the users can see the difference between two points in time. As an example, figure \ref{figureHFiguresSimple} shows two different samples for each measurement. This example portrays a modelled person that has been active in a health coach program. Some measurements have improved and are closer to the recommendation. Users repeatedly expressed that it would be very helpful to visualize two or more different points in time so as to compare how the person has evolved. 

Including multiple graphs has implications in the structure and procedures of the visualization construction. For instance, we structured the SVG document such that each measurement includes one or many plotted circles that map to each sample. The measurement labels need to be positioned considering that a plotted circles can (and probably will) overlap. This is challenge that we address in the next section by finding an optimal label space distribution to avoid labels from overlapping and also to reduce the clutter in the visualization space.

\subsection*{Label Space Distribution}

After the measurements are plotted, the labels are added to increase readability. The position of the label needs to be defined within a given range to avoid overlaps and clutter.

% Mention the margin can be specified also by the user

Labels need to avoid overlapping with each other and with their measurements. To solve the label overlapping problem we implemented an algorithm that starts by ordering the labels by their angular position, that is the angle at which the measurement is positioned. The next step is to calculate the height of the label and position it over the previous one in the direction that goes from the center of the visualization area upwards or downwards (depending on the angle). The idea is to begin with the center of the area, either to the left or to the right of the circles, then we work our way up or down drawing the labels into the SVG document. We add the labels as SVG elements and the use the \texttt{transform} property to position them in the corresponding place. Figure \ref{figureVerticalLabelDistribution} shows the spacing between the labels using the algorithm when drawing the labels from the center to the upper right corner. For each of the four quadrants, the library calls the method shown in figure \ref{codeVerticalLabelDistribution} which computed the position of the label as we described.

As mentioned before, labels can also overlap with measurement circles. To avoid this problem we calculate the maximum radius from the center of the visualization area to the highest value of a measurement sample. From that starting point, we place the label in that position. In other words, for each measurement, we find the largest value of the samples. Figure \ref{figureRadiusMeasurementDistrbution} shows a sector of the hFigures where the sugar measurement label has been pushed out for a few pixels in order to avoid overlapping it with the red circle. The rest of the labels adjust to that position by leaving a user-defined margin.

\begin{equation} \label{eq:maxRadius}
	r_{label} = max(\bigcup_{i}^{n} \left \{ r_{i} \right \} ) + \textit{margin}
\end{equation}

The radius for the label is the maximum value of the samples translated as graphical coordinates plus a margin. Equation \ref{eq:maxRadius} obtains the label radius $r_{label}$ given the radii of the samples of a measurement plus the default margin $m$.

\subsection*{User-centred Design}

% IMPORTANT Specify the ISO!

% HANNU: I would leave out the sentence about the expertise areas of the users. 
% Andres: While it is true that we had a mixed audience in our design process, I think you are right. We should try to simplify the idea of the user-centered design in the paper.
% Andres: Now reviewers are asking for more detail so I will add this with the methods we used, which was onl the interview type because we demonstrated the prototype and gathered the feedback from the research group.
% From the expertise perspective we had a medical doctor, signal processing, usability, data science, machine learning and software engineering feedback.

The users that participated in the design of the library have several areas of expertise including : Medical Sciences, Signal Processing, Machine Learning, Statistics, Software Engineering and User Experience.  The user-centred approach provided us timely feedback to decide on the features that are more important to the intended users. Table \ref{tbl:features} lists the supported features of hFigures compared to the hGraph. This section summarized the key differences and improvements of hFigures.

The key improvement of hFigures is the addition of multiple graphs as a mechanism to compare the values of the health measurements over time.

The immutable shape of the graph presents the same information (values of the measurements in respect to the recommended target) regardless of zooming. This feature shows the data ``as is'' without calculating average, mean or deviation. Users stressed the importance of graphically representing the information without any calculations such as mean or accumulated values. The users expressed that showing the measurement values in hFigures helped them to derive valuable insights with just a quick glance at the data, for instance they identified measurement that fall outside of the recommended range easier and without requiring them to zoom in or out.

The hFigures library does not calculate an overall score since the users considered that this task should be the sole responsibility of a health care professional. An overall score also depends on each person under a case by case basis. For instance, the hGraph allows the user to assign weights to each measurements' value, the score is then calculated summing the value of the weights times the measurement's deviation from the recommendation. The users participating in the design process of our library expressed that an overall score would complicate the integration of the library into daily health care processes as specialists would need to review case by case to find the adequate score formula, which means specifying the weights (importance) of each measurement for a given person.

Measurements in the hFigures library are grouped in sectors which represent the category they belong to. Grouped measurements allow a clear division of categories resulting in enhanced understanding on how certain areas of wellness have changed and how, if any, they affect each other. The sectors remain visible regardless of the zoom level, users expressed that this feature provides an informative approach as the categories are always showing to which category the measurements belong to. Hiding the category labels and displaying the measurements without divisions would complicate understanding the status of health categories, such as sleep, nutrition, physical activity and others. Users expressed their confusion when they were unable to determine when a category starts or ends after zooming in and out of the hGraph.

The possibility to export the generated figure as an SVG file, allows the integration into research articles, presentations, websites, posters and other Software applications to further enhance the utility of the hFigures.

\section*{Evaluation}

The health data visualization library was placed in the context of a full application. We tested the library in a contextualized scenario where the users conducted a series of tasks and answered usability questionnaires. In this section we present the methods we used for recruiting the participants and for the usability testing of the library. We also explain the metrics measured and the rationale behind the selection of the usability testing methods.

Nielsen suggests that ```usability has multiple components and is traditionally associated with the five usability attributes, which are learnability, efficiency, memorability, errors, and satisfaction'' \cite{nielsen1994usability}. In order to assess the usability of the software solution, multiple alternatives exist in industry and research. Johnson \textit{et al}. developed a toolkit for usability testing of Electronic Health Records commissioned by the Agency for Healthcare Research and Quality of the U.S. Department of Health and Human Services \cite{johnson2011ehr}. The toolkit is built on the basis of the assessment of existing usability methods in the context of Electronic Health Records and Health Information systems. The toolkit is a detailed analysis of the usability methods, their advantages, disadvantages and appropriateness ranking.

We selected the Usability Questionnaires since it has a high appropriateness ranking \cite{johnson2011ehr}. We were able to recruit three usability experts to conduct the Heuristic Evaluation and the Cognitive Walkthrough, both are recommended techniques to complement the evaluation. We concluded the evaluation with Controlled User Testing since we had a laboratory and the necessary tools to conduct the tests.

We conducted usability testing on an application with the hFigures as a visualization component of the overall health situation of a modelled patient. The testing process started with the signature of an informed consent where we explained the participants the purpose of the test. Afterwards we proceed to explain a usability scenario and the tasks that the participants were asked to complete. The participants were allowed to ask questions at any time. After performing the tasks we asked the participants to fill in the Usability Questionnaires. We close the session with a briefing interview where we asked the users what they liked and disliked about the application as well as what were their recommendations for further improvements.

\subsection*{Continuous Health Monitoring Application}

In order to test the hFigures library, we designed an application for visualizing the health situation of a modelled patient and how this has changed over time within health coaching program. The objective is to help the users in the decision making process of assessing the overall health situation and weather or not the health program has provided benefits.

The application has three components: activity timeline, the hFigures data visualization library and longitudinal measurements.

\paragraph*{Activity Timeline} This component represents the health interventions (particular actions) that the modelled patient has done during the health coaching program. During the program, several snapshots of the patient's overall health are taken and visualized using the hFigures library.

\paragraph*{hFigures} The set of measurements taken during the health coaching program is a holistic view of the health situation of the modelled patient. This is suitable to be displayed using the hFigures library which we present in the application. Users can change the time at which the snapshot was taken to compare changes over time as a result of the health interventions.

\paragraph*{Longitudinal Measurements} The application also displays the same set of measurements using longitudinal representation. We included this component to provide additional details on how measurements have changed over the coaching program.

\subsection*{Recruitment}

How many according to Falk and for Expert Users how many according to Nielsen.

We recruited...
How we recruited them...
What we gave in return...
Informed consent signed...

\subsection*{Heuristic Evaluation}

Heuristic Evaluation requires at least one expert in the area of human-computer interaction \cite{nielsen1994usability, johnson2011ehr}. For our evaluation we recruited three experts, they assessed the application using Nielsen's heuristics \cite{nielsen1994usability}. The evaluation has 11 metrics evaluated using a seven point Likert scale, where the value 1 indicates ``strongly disagree'' and 7 ``strongly agree''

% REF here!
% http://www.nngroup.com/articles/ten-usability-heuristics/

Heuristics are ``rules of thumb'' comprised of 10 principles meant to assist the Human-Computer Interaction specialist in the usability assessment \cite{johnson2011ehr, Nielsen10UsabilityHeuristics}. We explain the principles of the Heuristic Evaluation according Nielsen \cite{Nielsen10UsabilityHeuristics}.
\begin{enumerate}
\item \textit{Visibility of the System Status}: Refers to continuous feedback on the status of the system ``within reasonable time'' (Feedback).
\item \textit{Match between system and the real world}: The use of the language should be familiar to the user so that conversations follow a ``natural and logical order'' avoiding technical terminology unfamiliar to the intended user audience (Speak the User's Language).
\item \textit{User control and freedom} : Allow the user to recover from erroneous navigational options with ``clearly marked'' access options (Clearly Marked Exits).
\item \textit{Consistency and standards} : Follow the same language and terminology to avoid the user from guessing the meaning of ``words, situations, or actions''(Consistency).
\item \textit{Error prevention} : Avoid ``error-prone'' options in the system whenever possible and for those cases when the problematic options cannot be avoided, then present the user confirmation dialogues (Prevent Errors).
\item \textit{Recognition rather than recall} : Present visible options to the user at all times so as to avoid the effort of remembering previously stated instructions. Whenever options cannot be visible, make them ``easily retrievable whenever appropriate'' (Minimize User Memory Load).
\item \textit{Flexibility and efficiency of use} : The interface should accommodate the novice and advance user by providing ``tailored frequent actions'' (Shortcuts).
\item \textit{Aesthetic and minimalist design} : The dialogues should only contain relevant and clear information that is timely needed at that particular state of the interface (Simple and Natural Dialogue).
\item \textit{Help Users recognize, diagnose, and recover from errors} : Plain language should be used in error messages, and whenever possible they should provide helpful information so that the users can take constructive actions. (Good Error Messages)
\item Help and documentation : In some, systems require documentation and guidelines to explain briefly how to accomplish specific tasks in ``concrete steps''.
\end{enumerate}

\subsection*{Cognitive Walkthrough}

Wharton \textit{et al}. developed the Cognitive Walkthrough for usability testing \cite{wharton1994cognitive}. Johnson \textit{et al}. summarize this method as a ``usability inspection method that compares the users’ and designers’ conceptual model and can identify numerous problems within an interface'' \cite{johnson2011ehr, wharton1994cognitive}.

Cognitive Walkthrough has successfully been used to evaluate usability of Healthcare Information Systems \cite{johnson2011ehr, peute2007significance, karahoca2010information, cohen2004cognitive, khajouei2009usability} and Web Information Systems \cite{blackmon2002cognitive}.

Since Cognitive Walkthroughs ``tend to find more severe problems'' \cite{sears1997heuristic, johnson2011ehr} but ``fewer problems than a Heuristic Evaluation'' \cite{beuscart2007human, johnson2011ehr} we included both methods in our evaluation.

\subsection*{Laboratory Testing}

Regarded as the ``golden standard'' for usability testing \citep{newman199810}, Laboratory Testing collects ``qualitative and quantitative'' data ``since it collects both objective data such as performance metrics (e.g., time to accomplish the task, number of key strokes, errors, and severity of errors) and subjective data such as the vocalizations of users thinking aloud as they work through representative tasks or scenarios'' \cite{johnson2011ehr}. 

Controlled user testing is comprised of ``a series of commonly used task scenarios'' where users are asked to conduct these tasks using a ``thinking aloud'' \cite{johnson2011ehr, ericsson1980verbal, nielsen1994usability}. This process requires ``users to talk aloud about what they are doing and thinking'' while they complete the tasks using the system \cite{johnson2011ehr, ericsson1980verbal, nielsen1994usability}.

As the ``golden standard'' in usability testing, this method has been widely used in evaluating Health Information Systems \cite{johnson2011ehr, borycki2009novice, currie2003clinical, hasman2006development, wu2008usability}

The data exploration tasks are designed to assist the decision making process on the health situation of the modelled patient. The list of tasks is structured as follows:
\begin{enumerate}
\item How many areas of health are displayed in the hFigures?
\item Choose one of these areas and point to its measurements.
\item Identify one measurement inside the recommended values and another one outside.
\item Identify the measurement that is the furthest from the recommended values.
\item What does the green, yellow and red circles mean?
\item Has the overall health improved after coaching?
\item Which area of health has improved the most after health coaching?
\item Which measurements show the biggest improvement?
\item Understand the difference between the points inside and outside the hFigures.
\end{enumerate}

\subsection*{Usability Questionnaires}

In addition the the previous methods we also used standardized questionnaires for two main reasons. Firstly, they allow users to express their feedback using validated questions. Secondly, they help us identify problems with the usability of the system.

We used three standard questionnaires in order to obtain additional information about the usability of the library. The questionnaires we used are Computer System Usability Questionnaire (CSUQ), After Scenario Questionnaire and Usefullness (ASQU), Satisfaction and Ease-of-Use Questionnaire (USE). All three questionnaires use a seven point Likert scale.

Table \ref{tbl:questionnaires} shows the length, reliability and the metrics of the questionnaires. 

\subsubsection*{Computer System Usability Questionnaire} Explain and cite again.

\subsubsection*{After Scenario Questionnaire} Explain and cite again.

\subsubsection*{Usefulness, Satisfaction and Ease of Use Questionnaire} Explain and cite again.


\subsection*{Data Model}

Similar to the study we conducted in the insight-based methodology \citep{EMBCLedNiem}, we modeled a patient using clinical expertise of a physician along with the most common symptoms for developing Type II Diabetes. The modelled patient consisted of a set of measurements over time comprised of the following parameters:

\begin{itemize}
	\item Blood pressure: systolic and diastolic blood pressure 
    \item Physical activity: weekly active days\cite{us2008physical, physical2008physical}, steps per day \cite{tudor2004many} 
    \item Body composition: Body Mass Index (BMI), waist diameter and fat percentage
    \item Sleep: time in bed, time asleep  
    \item Fitness: resting heart rate, fitness index \cite{oja2013tester,laukkanen1992validity}, muscular force, muscular endurance and balance \cite{suni2009fitness}
    \item Lab Tests: hemoglobin, fB-Gluc, cholesterol, HDL, LDL, triglycerides 
    \item Nutrition: meal regularity, type of meals (vegetables and fruits), sugar intake, fat quality, fiber and salt intake.
    \item Drugs: tobacco (cigarettes per day), alcohol abuse, drug abuse (narcotics), medication abuse
    \item Emotional wellbeing: depression level \cite{poutanen2010validity}, stress level and stress recovery \cite{firstbeat2014, teisala2014associations} and optimism \cite{scheier1994distinguishing}.
\end{itemize}	


\section*{Results}

\subsection*{Heuristic Evaluation}

The three expert users answered the Heuristic Questionnaire in order to identify problems with the user interface of the health monitoring application. The three experts agreed and in some cases strongly agreed most of the indicators. One expert found the instructions for adjusting the time of the visualization tool to be demanding. The expert addressed this comment to the integration interface that allows the time to be adjusted and thus visualized. The remark was not addressed to the graphical representation of the data using hFigures. The results of the evaluation are summarized in table \ref{tbl:heuristicsResults}. The average response was 6.3 out of 7 points.


\subsection*{Cognitive Walkthrough}

% Note : when refencing a source with more than three authors we use et al. 
% The notation has a dot at the end of 'al' and both words are italliced :
% http://blog.apastyle.org/apastyle/2011/02/et-al-when-and-how.html

During the Cognitive Walkthrough, the concept of the health monitoring application was explained to the usability experts. The purpose of the application was explained in the context of the health situation of the modelled patient and how the application visualizes the changes in the health situation over time. We used the usability scenario and tasks to confirm that the interface supports the intended use of the application. The questions comprising the walkthrough, as described by Wharton \textit{et al} \cite{wharton1994cognitive}., were correctly answered by the expert users thus no design or mismatch errors were found.

\subsection*{Controlled User Testing}

The Controlled User Testing allowed us to find the most problematic tasks. The results are summarized in table \ref{tbl:tasksResults}. The table \ref{tbl:tasksResults} shows both, crucial or non-crucial errors as a total. Tasks that were not completed by all 14 participants are those in which crucial errors occurred.

A total of 7 out of the 9 tasks were successfully completed by all 14 participants. The most problematic task was to find the measurements in the area of health (task 2) with 3 out of the 14 participants unable to complete the task. Difficulties with this task were caused by a maladjustment in the zoom default value of the library. This bug was corrected after we concluded the full usability testing.

The tests show that four users erroneously identified the areas of health (task 1). We counted two errors respectively in identifying the measurement with the furthest deviation from the recommendation (task 4) and determining at least one measurement that improved after the coaching program (task 6). A single error occurred in identifying a measurement inside and another outside the recommendation (task 3), determining the measurement that improved the most after the coaching(task 8) and understanding the difference between the points inside and outside the recommendation (task 9). 

\subsection*{Usability Questionnaires}

We present the results of the CSUQ divided by sections. The first section is the evaluates the System Usefulness and the results are presented in table \ref{tbl:CSUQSystemUsefulnessResults}.

\subsection*{Identified Issues and Suggested Improvements}

\section*{Discussion}

The value of a data visualization depends on the knowledge that it can convey to the public. In this section, we claim that hFigures has the potential to be used both in the clinical and personal wellness applications. Large amounts of measurements do not clutter the visualization area as a result of our implementation, figure \ref{figureHFiguresComplex} shows an example of a complex holistic visualization of a modelled patient. The comparison of multiple graphs can provide a meaningful visualization to individuals and clinicians alike. The implementation of hFigures follows an extensible approach and even though it was designed to be used for health data visualization, any dataset that has a target range of values as a reference, can be visualized.

\subsection*{Further Development}

We plan to develop the library further to address the suggestions obtained from the participants of the usability testing.

\subsection*{Translation to Health Care}

Nowadays, the formats of EHR and application data follow a large variety of formats. It remains an open challenge to incorporate these data sources into a holistic repository that portrays a patient using every possible measurement, from laboratory tests to activity trackers. Currently, most of the data sources contain the date when the sample was extracted, whether it is a tracking device or a blood test. This sample date already provides the timestamp required by the hFigures data source file. The values of the samples are the main object of study in a measurement, for instance the levels of cholesterol or sugar at a given time, the number of steps per day, the percentage of body fat and several others. This information can be transformed in a simple process to build the hFigures data source following the JSON structure.

hFigures is a visualization library based on Web technologies, it uses a Web browser and the rendering of the SVG is compatible with current HTML standards. Thus providing interoperability across multiple devices including tablets, smartphones, workstations or laptops is indeed feasible. The D3.js library that hFigures is built on, enables compatibility with Internet Explorer versions 8 and higher using a component named \textit{Aight} \cite{aight}. Internet Explorer 8 is prevalent in hospitals and clinics due to the restrictions in installing custom software.

\subsubsection*{Patient Evolution}

Multiple graphs plotted on the layout of recommended values shows the change in the data over time. This could allow clinicians to understand the evolution of certain aspects in the health of the patient. For instance, health professionals would be able to look at the effect of a trail drug treatment over time. Possibly, the collected samples of a patient before and after starting the trial would be plotted as the multiple graphs portraying the evolution of the patient. Figure \ref{figureHFiguresComplex} shows an example of a more complex dataset. The labels for each individual measurement are usually hidden when a full zoom out is performed by the user. For the purpose of demonstrating the visualization library we have made all the labels visible. 


\subsubsection*{Personal Health Monitoring}

As an example, Fitbit provides activity trackers and a wide-range of devices. The data collected can be obtained through their API. In most cases, providers such as Fitbit follow an HTTP REST interface. Figure \ref{codeFitbitApi} shows an example of the data Fitbit provides through its interface. The data has \texttt{"activities"} as properties of a JavaScript object. These entries have in turn a property \texttt{"startTime"} which provides the timestamp required for the hFigures data source file.

In this case, a step counter contains the property \texttt{"steps"} with the number of steps registered by the device. This and other measurement can be plotted in the hFigures following the data source file structure.

We can also use multiple graphs to show the user-defined goals as a reference in addition to the actual values of the measurement (steps per day). Figure \ref{codeFitbitApi} only show steps per day, however the documentation of the API specifies that additional data is available. This data includes distance travelled, sedentary activity, floors climbed, calories burnt and more.

Other device manufacturers such as Withings or Jawbone provide their users the possibility to use their APIs to extract collected data in a similar way. As in the case of Fitbit, all data has at least a timestamp (date, time or both) and a set of values.

In order to specify the recommended range for these measurements in the hFigures data source file, it is possible to use websites such as WebMD or Mayo Clinic to obtain their recommendation for such values.

\subsection*{Future Work}

We plan to add features to the library and use it in future studies. Animation effects and other graphical elements will be incorporated so as to facilitate the understanding of how the measurements have changed over time. A potential addition to the library includes persuasive design elements to enhance the attitudes towards behavior change.

\section*{Conclusion}

A user-centered approach was followed to develop a health data visualization library, hFigures. It is based on the design idea of the hGraph but provides additional features. The goal is to enhance users' understanding of health data.
By comparing multiple graphs, users get a graphical representation of the overall evolution of health measurements with respect to the recommended targets. The label spacing algorithm reduces clutter and visually separates the data into adequate spaces. Groups are clearly divided in the visualization resulting in reduced clutter. This helps users to focus on separate regions of the dataset. The hFigures library also generates an SVG structure embedded in the HTML document. The SVG can be extracted and can be directly utilized as an image file for illustrations in scientific articles and as input for other software applications.

hFigures is an open source library that aims to improve the understanding of health data through visualization techniques. The goal is to enhance individuals to take informed decisions in order to change their health behaviors towards a healthier society.


\section*{Availability and Requirements}

\begin{itemize}
\item \textbf{Project name:} hFigures
\item \textbf{Project home page and source code repository:} \url{https://github.com/ledancs/hFigures}
\item \textbf{Operating System:} Platform independent.
\item \textbf{Programming language:} JavaScript.
\item \textbf{Other requirements:} Developers willing to deploy the application need to serve the files via a Web server. Users require a Web browser to visualize the application.
\item \textbf{License:} Text for this section.
\item \textbf{Any restrictions to use by non-academics:} No.
\end{itemize}

%%%%%%%%%%%%%%%%%%%%%%%%%%%%%%%%%%%%%%%%%%%%%%
%%                                          %%
%% Backmatter begins here                   %%
%%                                          %%
%%%%%%%%%%%%%%%%%%%%%%%%%%%%%%%%%%%%%%%%%%%%%%

\begin{backmatter}

\section*{Abbreviations}
    hGraph: Health graph; hFigures: Health figures; JSON: JavaScript object notation; EHR: Electronic health record; PHR: Personal health record; HTML: Hypertext markup language; SVG: Scalable vector graphics; D3: Data-driven documents; API: Application programming interface.

\section*{Competing Interests}
    The authors declare that they have no competing interests.

\section*{Author's Contributions}
AL developed the implementation of the hFigures library using a user-centered approach. AL and HN wrote jointly the article. HN suggested several use cases that helped shaped the design process. HN contributed in the design and conception of the library. AL, HN and MAM designed the application for health coaching. AL and MAM implemented the application. AL integrated the application. MAM conducted the user testing and analysed the results.
    
\section*{Acknowledgements}
This research was supported jointly by TEKES (the Finnish Funding Agency for Technology and Innovation) as part of the Digital Health Revolution, as well as the European Commission and TEKES under the ARTEMIS-JU WithMe project.
  
\section*{Author's Information}
AL and HN are part of the Personal Health Informatics research group from the Department of Signal Processing of Tampere University of Technology. AL is a Software Engineer pursuing PhD studies in Health Data Visualization with a strong background in Web Engineering and Health Information Systems. HN is a PhD senior researcher with background in biomedical engineering, signal processing, electrical engineering, user interface design and service design.

%%%%%%%%%%%%%%%%%%%%%%%%%%%%%%%%%%%%%%%%%%%%%%%%%%%%%%%%%%%%%
%%                  The Bibliography                       %%
%%                                                         %%
%%  Bmc_mathpys.bst  will be used to                       %%
%%  create a .BBL file for submission.                     %%
%%  After submission of the .TEX file,                     %%
%%  you will be prompted to submit your .BBL file.         %%
%%                                                         %%
%%                                                         %%
%%  Note that the displayed Bibliography will not          %%
%%  necessarily be rendered by Latex exactly as specified  %%
%%  in the online Instructions for Authors.                %%
%%                                                         %%
%%%%%%%%%%%%%%%%%%%%%%%%%%%%%%%%%%%%%%%%%%%%%%%%%%%%%%%%%%%%%

% if your bibliography is in bibtex format, use those commands:
\bibliographystyle{bmc-mathphys} % Style BST file (bmc-mathphys, vancouver, spbasic).
\bibliography{bmc_article}      % Bibliography file (usually '*.bib' )

% or include bibliography directly:
% \begin{thebibliography}
% \bibitem{b1}
% \end{thebibliography}

%%%%%%%%%%%%%%%%%%%%%%%%%%%%%%%%%%%
%%                               %%
%% Figures                       %%
%%                               %%
%% NB: this is for captions and  %%
%% Titles. All graphics must be  %%
%% submitted separately and NOT  %%
%% included in the Tex document  %%
%%                               %%
%%%%%%%%%%%%%%%%%%%%%%%%%%%%%%%%%%%

%%
%% Do not use \listoffigures as most will included as separate files

\section*{Figures}

%%%%%%%%%%%%%%%%
% Code figures %
%%%%%%%%%%%%%%%%

% Data source
\begin{figure}[h!]
  \caption{\csentence{JSON data source file.} The data source file structured as a JSON file.}
  \label{codeDataSource}
\end{figure}

% Layout construction
\begin{figure}[h!]
  \caption{\csentence{Layout construction code.} The layout for distributing the measurements is build using d3 pie layout leaving spaced between the measurement groups.}
  \label{codeLayoutConstruction}
\end{figure}

% fitbit API
\begin{figure}[h!]
  \caption{\csentence{Fitbit API example response.} An example JSON response from a Fitbit activity sensor that can be transformed to a data source to be visualized by hFigures.}
  \label{codeFitbitApi}
\end{figure}

% Label positioning
\begin{figure}[h!]
  \caption{\csentence{JavaScript code for label positioning.} The function in JavaScript distributed the positioning of the label to avoid overlapping and clutter.}
  \label{codeVerticalLabelDistribution}
\end{figure}

% Color-coded entries
\begin{figure}[h!]
  \caption{\csentence{JavaScript code to determine if additional ranges are provided.} In JavaScript the data source could contain additional ranges, these are properties in an object that need to be checked beforehand and if the exists, compare the values accordingly.}
  \label{codeColorCoded}
\end{figure}

%%%%%%%%%%%%%%%%%%
% Visualizations %
%%%%%%%%%%%%%%%%%%

% Sonin's hGraph
\begin{figure}[h!]
  \caption{\csentence{An example of the hGraph.} The hGraph shows an overview at a low zoom level. When the zoom increases, the details of each measurement are revealed.}
  \label{figureHGraphZoom}
\end{figure}

% hFigures simple example
\begin{figure}[h!]
  \caption{\csentence{A simple example of hFigures.} The measurements are separated by groups in sectors.}
  \label{figureHFiguresSimple}
\end{figure}

% Radius and measurement labels
\begin{figure}[h!]
  \caption{\csentence{Label positioning avoiding overlapping with measurements.} Some measurements can be positioned outside of the recommended range, the labels are adjusted to avoid overlapping.}
  \label{figureRadiusMeasurementDistrbution}
\end{figure}

% Label positioning
\begin{figure}[h!]
  \caption{\csentence{Label positioning distribution.} The labels are positioned according the their height and margin, as a result the labels do not overlap and clutter is avoided.}
  \label{figureVerticalLabelDistribution}
\end{figure}

% hFigures complex
\begin{figure}[h!]
  \caption{\csentence{A complex hFigures example.} A holistic health assessment of a modelled person comprised of several measurements with two time snapshots showing the person's evolution.}
  \label{figureHFiguresComplex}
\end{figure}

%%%%%%%%%%%%%%%%%%%%%%%%%%%%%%%%%%%
%%                               %%
%% Tables                        %%
%%                               %%
%%%%%%%%%%%%%%%%%%%%%%%%%%%%%%%%%%%

%% Use of \listoftables is discouraged.
%%
\section*{Tables}

\begin{table}[h!]
\caption{Comparative Functionality Table. The table lists the features supported by hFigures and hGraph.}
      \begin{tabular}{lcc}
        \hline
        Functionality                            & hFigures  & hGraph    \\ \hline
        Color-coded measurements                 & \ding{51} & Partially \\ 
        Recommended area                         & \ding{51} & \ding{51} \\ 
        Circular distribution of measurements    & \ding{51} & \ding{51} \\ 
        Plot values at specific times            & \ding{51} & ---       \\ 
        Time series data source                  & \ding{51} & ---       \\ 
        Immutable shape                          & \ding{51} & ---       \\ 
        Weighted score                           & ---       & \ding{51} \\ 
     		Multiple graphs                          & \ding{51} & ---       \\
    		Visible category labels                  & \ding{51} & ---       \\
    		Label space distribution                 & \ding{51} & Partially \\
    		Grouped measurements                     & \ding{51} & ---       \\
    		SVG export                               & \ding{51} & ---        \\ \hline
		
      \end{tabular}
\label{tbl:features}
\end{table}

\begin{table}[h!]
\caption{Standard Questionnaires Table. The table lists the standard questionnaires we used for the user evaluation of the system with their length, reliability and metrics.}
      \begin{tabular}{lccl}
        \hline
        Questionnaire & Questions & Reliability & Metrics \\ \hline
        \multirow{4}{*}{CSUQ} & \multirow{4}{*}{19} & 0.93 & Usefulness \\
         & & 0.91 & Information Quality \\
    		 & & 0.89 & Interface Quality \\
    		 & & 0.95 & Overall Usability \\ \hline
        \multirow{3}{*}{ASQ} & \multirow{3}{*}{3} & \multirow{3}{*}{0.93} & Efficiency \\
         & & & Effectiveness \\
         & & & Satisfaction \\ \hline
		
      \end{tabular}
\label{tbl:questionnaires}
\end{table}

\begin{table}[h!]
\caption{Heuristic Evaluation Results. The table summarizes the results of the Heuristic Evaluation conducted by three usability experts.}
      \begin{tabular}{p{3.85cm}R{1.2cm}R{1.2cm}}
      \hline
      Heuristic & Average Response & Standard Deviation \\ \hline
      Visibility of system status & 6.00 & 1.00 \\
      Match between system and the real world & 6.33 & 0.577 \\
      User control and freedom & 6.33 & 0.57 \\
      Consistency and standards & 6.67 & 0.57 \\
      Error prevention & 6.33 & 0.57 \\
      Recognition rather than recall & 4.67 & 0.57 \\
      Flexibility and efficiency of use & 6.67 & 0.57 \\
      Aesthetic and minimalist design & 7.00 & 0.00 \\
      Help users recognize, diagnose, and recover from errors & 6.33 & 0.57 \\
      Help and documentation & 6.67 & 0.57 \\ \hline
      Nielsen heuristic evaluation & 6.30 & .562 \\ \hline
      \end{tabular}
\label{tbl:heuristicsResults}
\end{table}

\begin{table}[h!]
\caption{Controlled User Testing Results. The table summarizes the results of the 14 users performing the 9 tasks.}
      \begin{tabular}{lC{1.5cm}C{0.75cm}R{1.2cm}R{1.2cm}}
      \hline
      Task & Successfully Completed & Errors & Average Time (seconds) & Standard Deviation (seconds)\\ \hline
      Task 1 & 14 & 4 & 12.2 & 12.6 \\
      Task 2 & 11 & 0 & 10.0 & 12.4 \\
      Task 3 & 14 & 1 & 10.8 & 5.4 \\
      Task 4 & 14 & 2 & 6.8 & 5.0 \\
      Task 5 & 14 & 0 & 17.5 & 6.6 \\ 
      Task 6 & 14 & 0 & 16.1 & 16.5 \\ 
      Task 7 & 14 & 0 & 6.2 & 5.6 \\ 
      Task 8 & 14 & 1 & 7.9 & 5.9 \\ 
      Task 9 & 13 & 1 & 9.2 & 4.7 \\ \hline
      \end{tabular}
\label{tbl:tasksResults}
\end{table}

\begin{table}[h!]
\caption{Computer System Usability Questionnaire Results for the System Usefulness assessment. The table shows the results of the questions corresponding to the System Usefulness with its average and standard deviation.}
      \begin{tabular}{L{4cm}R{1.2cm}R{1.2cm}}
      \hline
      Question & Average Time (seconds) & Standard Deviation (seconds)\\ \hline
      Overall, I am satisfied with how easy it is to use this system & 6.29 & 0.99 \\
      It was simple to use this system & 6.07 & 1.20 \\
      I can effectively complete my work using this system & 6.07 & 1.07 \\
      I am able to complete my work quickly using this system & 5.86 & 1.40 \\
      I am able to efficiently complete my work using this system & 6.21 & 0.89 \\
      I feel comfortable using this system & 6.21 & 0.97 \\
      It was easy to learn to use this system & 6.43 & 0.85 \\
      I believe I became productive quickly using this system & 5.93 & 1.26 \\ \hline
      System Usefulness & 6.13 & 0.93 \\ \hline
      \end{tabular}
\label{tbl:CSUQSystemUsefulnessResults}
\end{table}

\begin{table}[h!]
\caption{Computer System Usability Questionnaire Results for Information Quality.}
      \begin{tabular}{L{4cm}R{1.2cm}R{1.2cm}}
      \hline
      Question & Average Time (seconds) & Standard Deviation (seconds)\\ \hline
      The system gives error messages that clearly tell me how to fix problems & 4.50 & 2.44 \\
      Whenever I make a mistake using the system, I recover easily and quickly & 5.43 & 1.95 \\
      The information (such as online help, on-screen mes-sages, and other documentation) provided     with this system is clear & 5.29 & 1.90 \\
      It is easy to find the information I needed & 6.07 & 1.27 \\
      The information provided for the system is easy to un-derstand &  5.93 & 1.39 \\
      The information is effective in helping me complete the tasks and scenarios & 6.14 & 1.17 \\
      The organization of information on the system screens is clear & 6.29 & 1.14 \\ \hline
      Information Quality & 5.66 & 1.20 \\ \hline
      \end{tabular}
\label{tbl:CSUQInformationQualityResults}
\end{table}

\begin{table}[h!]
\caption{Computer System Usability Questionnaire Results for Interface Quality.}
      \begin{tabular}{L{4cm}R{1.2cm}R{1.2cm}}
      \hline
      Question & Average Time (seconds) & Standard Deviation (seconds)\\ \hline
      The interface of this system is pleasant & 6.36 & 1.00 \\
      I like using the interface of this system & 6.36 & .929 \\
      This system has all the functions and capabilities I expect it to have & 6.00 & 1.18 \\ \hline
      Interface Quality & 6.24 & .999 \\ \hline
      \end{tabular}
\label{tbl:CSUQInterfaceQualityResults}
\end{table}

\begin{table}[h!]
\caption{Computer System Usability Questionnaire Results for Overall Usability, System Usefulness, Information and Interface Quality.}
      \begin{tabular}{L{2.5cm}lR{1.2cm}R{1.2cm}}
      \hline
      Metric & Questions & Average Time (seconds) & Standard Deviation (seconds)\\ \hline
  		Overall Usability   &  1-19 & 6.02 & 1.04 \\ 
      System Usefulness   &  1-8  & 6.13 & 0.93 \\ 
      Information Quality &  9-15 & 5.66 & 1.20 \\ 
      Interface Quality   & 16-18 & 6.24 & .999 \\ \hline
      \end{tabular}
\label{tbl:CSUQOverallUsabilityResults}
\end{table}

\begin{table}[h!]
\caption{After Scenario Questionnaire Results.}
      \begin{tabular}{L{4cm}R{1.2cm}R{1.2cm}}
      \hline
      Question & Average Time (seconds) & Standard Deviation (seconds)\\ \hline
      Overall, I am satisfied with the ease of completing the tasks in this scenario & 6.64 & .842 \\
      Overall, I am satisfied with the amount of time it took to complete the tasks in this scenario & 6.64 & .497 \\
      Overall Satisfaction of the system & 6.46 & .531 \\ \hline
      \end{tabular}
\label{tbl:ASQResults}
\end{table}


%%%%%%%%%%%%%%%%%%%%%%%%%%%%%%%%%%%
%%                               %%
%% Additional Files              %%
%%                               %%
%%%%%%%%%%%%%%%%%%%%%%%%%%%%%%%%%%%

%\section*{Additional Files}
%  \subsection*{hFigures}
%    Additional file descriptions text (including details of how to
%    view the file, if it is in a non-standard format or the file extension).  This might
%    refer to a multi-page table or a figure.
%
%  \subsection*{Additional file 2 --- Sample additional file title}
%    Additional file descriptions text.


\end{backmatter}
\end{document}
